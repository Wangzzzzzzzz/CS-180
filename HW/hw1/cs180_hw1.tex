\documentclass[11pt]{amsart}

%%%%%%%%%%%%%%%%%%%%%%%%%%%%%%%%%%%%%%%%
% Packages
\usepackage{amsmath}
\usepackage{amsthm}
\usepackage{amssymb}
\usepackage{enumerate}
\usepackage{exercise}
\usepackage{float}

\restylefloat{table}
\theoremstyle{theorem}
\newtheorem{exercise}{Exercise}


%%%%%%%%%%%%%%%%%%%%%%%%%%%%%%%%%%%%%%%%
%Math Macros
\newcommand\N{\mathbb{N}}
\newcommand\Z{\mathbb{Z}}
\newcommand\Q{\mathbb{Q}}
\newcommand\R{\mathbb{R}}


%%%%%%%%%%%%%%%%%%%%%%%%%%%%%%%%%%%%%%%%
%homework macros
\newcommand\duedate{Oct 11, 2018} %Change this accordingly
\newcommand\homeworknumber{1} %Change this accordingly

\author{Wang, Zheng (404855295)}
%\email{your.email.address.here@ucla.edu}

\title{CS 180 Homework \#\homeworknumber
\\ Due \duedate}

%%%%%%%%%%%%%%%%%%%%%%%%%%%%%%%%%%%%%%%%

\begin{document}
\maketitle
	
\begin{exercise}
Exercise 2, Page 22
\end{exercise}
	
\begin{proof}
\textbf{True}\\

Suppose towards contardiction that there exists some stable matching $ S $ such that $ (m,w) \notin S$.\\\\

Since all of the man and women must be mathched, then there exists pair $ (m, w_2) \in S$, where $ w_2 $ is some woman lower in $ m $'s list than $ w $ (as $ w $ ranks first in $ m $'s list ).

Also, there exists pair $ (m_2, w)  \in S$, where $ m_2 $ is some man lower in $ w $'s list than $ m $ (as $ m $ ranks the first in $ w $'s list).\\\\

Thus, take the pairs $ (m,w') $ and $ (m', w) $. Since it is known that $ m $ prefers $ w $ to $ w'$ and $ w $ prefers $ m $ to $ m' $, this is an instance of instalbe match, a contradiction to the assumption that the matching is stable. $ \qedhere $
\end{proof}
\hfill
\begin{exercise}
Exercise 3, Page 22 
\end{exercise}

\begin{proof}
\textbf{There exist a case where there is no stable pair} \\\\
An example is when there is 3 slots. \\
Suppose $ A $ network has program $ a_1, a_2, a_3 $, each with rating $ 1,3,5 $; $ B $ network has program $ b_1, b_2 ,b_3 $, each with rating $ 2,4,6 $.\\\\
This is summarized in the following table:\begin{table}
	\begin{tabular}{|c|c|l|l|c|l|l|}
		\hline
		\multicolumn{1}{|l|}{\textbf{Network}} & \multicolumn{3}{c|}{\textbf{A}}                                            & \multicolumn{3}{c|}{\textbf{B}}                                            \\ \hline
		Program                                & \multicolumn{1}{l|}{$a_1$} & $a_2$                    & $a_3$                    & \multicolumn{1}{l|}{$b_1$} & $b_2$                    & $b_3$                    \\ \hline
		Rate                                   & 1                        & \multicolumn{1}{c|}{3} & \multicolumn{1}{c|}{5} & 2                        & \multicolumn{1}{c|}{4} & \multicolumn{1}{c|}{6} \\ \hline
	\end{tabular}
\label{Table 1}
\end{table}
\textit{Claim}: For all schedules $ T $ used by $ A $, There exists a schedule $ S $ for $ B $ such that $ B $ can win 2 out of 3 slots.

\textit{Proof:} Let $ b_3 $ compete against $ a_1 $, $ b_2 $ compete agianst $ a_2 $, and $ b_1 $ compete against $ a_3 $.\\

\textit{Claim}: For all schedules $ S' $ used by $ B $, There exists a schedule $ T' $ for $ A $ such that $ A $ can win 2 out of 3 slots.

\textit{Proof:} Let $ a_3 $ compete against $ b_2 $, $ a_2 $ compete agianst $ b_1 $, and $ a_1 $ compete against $ b_3 $.\\

Since both of the network has a potential to win 2 out of 3 slots, which ever network wins less than 2 slots can shift the scheduel to win 2 slots, and make the other network winning only 1 slot. But this then make the other network become capable of shifting its scheduel and win two slots. This cycle can continue forever\\\\
Thus there is no stable matching.
\end{proof}
\hfill
\begin{exercise}
Exercise 8, Page 27
\end{exercise}

\begin{proof}
There exists a set of preferenc list such that a switch would imporve the partner of a women who switched preferencs:\\
Suppose the list of all men is: $ (m_1, m_2, m_3) $ and algorithm choose man in this order, the list of all women is : $ (w_1, w_2, w_3) $, and $ w_2 $ is the women who shifts perference.\\
Suppose the perference list for man is:
\[
\begin{cases}
	m_1 : (w_2, w_1, w_3)\\
	m_2 : (w_2, w_3, w_1)\\
	m_3 : (w_3, w_2, w_1)
\end{cases}\]\\
and for women, the \textbf{true} preference list is :
\[\begin{cases}
w_1:(m_1,m_2,m_3)\\
w_2: (m_3,m_2,m_1)\\
w_3:(m_2, m_3, m_1)
\end{cases}
\]\\
Then before $ w $ shift her preference, the algorithm does the following:\\

In the first round, $ m_1 $ propose and get engaged to $ w_2 $; then $ m_2 $ propose to $ w_2 $ and get engaged, setting $ m_1 $ free; then $ m_3 $ propose to $ w_3 $ and get engaged. 

In the second round, $ m_1 $ propose to $ w_1 $ and get engaged.

This result in the pairing:\begin{align*}
m_1 &\longleftrightarrow w_1\\
m_2 &\longleftrightarrow w_2\\
m_3 &\longleftrightarrow w_3
\end{align*}\\
After $ w_2 $ shift preference list to $ (m_3,m_1,m_2) $, the algorithm does the following:

In the first round, $ m_1 $ propose to $ w_2 $ and get engaged; then $ m_2 $ propose to $ w_2 $ and get reject, so $ m_2 $ remains free; then $ m_3 $ propose to $ w_3 $ and get engaged.

In the second round, $ m_2 $ propose to $ w_3 $ and get engaged, setting $ m_3 $ free.

In the third round, $ m_3 $ propose to $ w_2 $ and get engaged, setting $ m_1 $ free.

In the fourth round, $ m_1 $ propose to $ w_1 $ and get engaged.

The resulting pairing is:
\begin{align*}
m_1 &\longleftrightarrow w_1\\
m_2 &\longleftrightarrow w_3\\
m_3 &\longleftrightarrow w_2\\
\end{align*}
Then, $ w_2 $ ends up with an improved partner.
\end{proof}
\hfill
\begin{exercise}
	Exercise 4, Page 67
\end{exercise}


The list in ascending order is:
\begin{align*}
g_1(n) &= 2^{\sqrt{\log n}}\\
g_3(n) &= n(\log n)^3\\
g_4(n) &= n^{4/3}\\
g_5(n) &= n^{\log n}\\
g_2(n) &= 2^n\\
g_7(n) &= 2^{n^2}\\
g_6(n) &=2^{2^n}
\end{align*}\\\\
\textbf{Exercise 5(a)}

\begin{proof}
\hfill\\
Let $ P(n) $ be the statment such that:\[ P(n): ``1+2+\cdots + n = \frac{n(n+1)}{2}" \]
$ P(1) $ says that $ 1 = ((1+1) \cdot 1)/2$, which is true.\\
Aussme that $ P(n)$ is true, then we have
\[
1 + 2 + \cdots + n + (n+1) = \frac{n(n+1)}{2}+ \frac{2n+2}{2} = \frac{(n+1)(n+2)}{2}
\]
Thus, $ P(n+1) $ holds. By induction, $ P(n) $ holds for all $ n \in \N $.
\end{proof}\hfill\\
\textbf{Exercise 5(b)}

\begin{proof}
Let $ P(n) $ be the statement such that:\[ P(n): ``1 \times 2 + 2\times 3 + 3\times 4 + \cdots + n(n+1) = \frac{n(n+1)(n+2)}{3}" \]
$ P(1) $ says that $ 1\times 2 = \frac{1\cdot(1+1)\cdot (1+2)}{3} $, which is true.\\
Assume that $ P(n) $ holds, then we have
\begin{align*}
1 \times 2 + 2\times 3 + \cdots + n(n+1) + (n+1)(n+2) =& \frac{n(n+1)(n+2)}{3} + \frac{3(n+1)(n+2)}{3} \\
&= \frac{(n+1)(n+2)(n+3)}{3}
\end{align*}
Thus, $ P(n+1) $ holds. By induction, $ P(n) $ holds for all $ n \in \N $.
\end{proof}\hfill\\\\\\\\\\
\textbf{Exercise 6}\hfill\\
Using the mega-small step used in class, we take $ m $-step as a mega-step.

In the class, we discussed and algorithm that set $ m  = \sqrt{n} $. However, it is obvious that as we try more mega-steps, The total number of tries increases. This is because we the number of tries after the first egg breaks at a perticualr mega-step remains constant.

Thus the lower bound for this algorithm is $ m $. In the worst case, the algorithm will run at about $ 2m $ time.

However, we could see that if we go $ m $ step in the first mega-step try, but then go up by only $ m-1 $ after the egg passes the first mega-step try, and $ m-2 $ after passing the second mega-step try...(In general we go up by $ m-k $ after passing the $ k^{th} $ mega-step try), then we would always endup with $ (m-k) + k  = m $ tries whenever the first egg breaks. Thus, this should improve the algorithm discussed in the class.\\

So, when there are 200 steps, using this idea we end up getting that $$ m + (m-1) + (m-2) + \cdots + 1 = \frac{m(m+1)}{2} = 200 $$ So we have to take 20 tries (since the excat solution is 19.5) to solve this problem. Comparing to the algorithm that takes $ m=\sqrt{n} $, which should takes about 28 tries, this is an improvment.\\

Thus for $ n $ steps, we generalize the equation for 200-step case and get this equation: $$ \frac{m(m+1)}{2} = n $$ It has solution $ \frac{\sqrt{1+8n}-1}{2} $. Therefore, in general, we will have to take $ \lceil \frac{\sqrt{1+8n}-1}{2} \rceil $ steps to find out the answer. With this general formula, we can now proof that our algorithm is better than the one we discussed in class:\[
\lim_{n \rightarrow \infty} \frac{T_{class}(n)}{T_{new}(n)} = \frac{2\sqrt{n}}{\frac{\sqrt{1+8n}-1}{2}} = \sqrt{2}
\]
Since this limit is larger than 1. The algorithm discussed in the class will run longer than our new algorithm at large $ n $, so this new algorithm will have better performance.
\end{document}